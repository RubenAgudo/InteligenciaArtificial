\documentclass[hidelinks]{scrreprt}
\usepackage[spanish]{babel}
\usepackage{listings}
\usepackage{classicthesis}
\usepackage{arsclassica}

% Title Page
\title{CLIPS Laboratorio 4}
\author{Rub\'{e}n Agudo \\
    Jonatan Galean \\
    I\~{n}igo Ochoa de Erive}


\begin{document}
\maketitle
\tableofcontents

\begin{abstract}
    Implementaci\'{o}n del laboratorio 4, con las reglas
    necesarias para hacer funcionar la c\'{a}mara
\end{abstract}

\chapter{Creaci\'{o}n del carrete de 24}
Para realizar este cambio, basta con cambiar en la \texttt{defclass CARRETE\_DE\_DOS}
en el slot \texttt{N\_FOTOS} el range de 0 2 a 0 24.

Es decir:
\begin{lstlisting} [showspaces=false, numbers=left, breaklines=true]
(defclass CARRETE_DE_24
    (is-a CARRETE)
    (role concrete)
    (single-slot N_FOTOS
        (type INTEGER)
        (range 0 24)
        (default 0)
        (create-accessor read-write)
    )
)
\end{lstlisting}

\chapter{Implementaci\'{o}n de la c\'{a}mara}
Para realizar la implementaci\'{o}n primero hemos pensado a ver cuales eran 
las restricciones que se aplican a la hora de realizar una foto de manera anal\'{o}gica.

Y pensamos que son las siguientes:
\begin{itemize}
    \item Que tenga un carrete puesto
    \begin{itemize}
        \item Que ese carrete no est\'{e} terminado
    \end{itemize}
    \item Que la tapa est\'{e} abierta
    \item Que el seguro est\'{e} quitado
    \item Que est\'{e} cargada (en nuestra opini\'{o}n esto
    es redundante, ya que si tiene un carrete puesto, y no est\'{a} terminado
    obviamente est\'{a} cargada.)
\end{itemize}

\section{Reglas necesarias}
Para realizar la siguiente funcionalidad implementamos las siguientes reglas.

\begin{lstlisting} [showspaces=false, numbers=left, breaklines=true]
(defrule sacarfoto
    (eq ABIERTA (send [miCamara] get-TAPA_CIERRA))
    (eq SI (send [miCamara] get-CARGADA))
    (eq QUITADO (send [miCamara] get-SEGURO))
    ?carrete <- (send [miCamara] get-PELICULA)
    (test < (send [miCamara] get-N_FOTOS) (send ?carrete get-N_FOTOS))
=>
    (send [miCamara] putt-N_FOTOS (+ 1 send [miCamara] get-N_FOTOS))
)
\end{lstlisting}

Pero entonces pensamos, \textquestiondown que sucede si la tapa est\'{a} cerrada, o el seguro quitado?

Para solventar el problema creamos las dos siguientes reglas:
\begin{lstlisting} [showspaces=false, numbers=left, breaklines=true]
(defrule quitarSeguro
    (eq PUESTO (send [miCamara] get-SEGURO))
=>
    (send [miCamara] putt-SEGURO QUITADO)
)

(defrule quitarTapa
    (eq CERRADA (send [miCamara] get-TAPA_CIERRE))
=>
    (send [miCamara] putt-TAPA_CIERE ABIERTA)
)
\end{lstlisting}

En este instante ya somos capaces de quitar el seguro y de abrir la tapa en caso de que estuviera cerrada.
Ahora s\'{o}lo queda cambiar el carrete si lo hemos terminado, pero para ello antes es necesario rebobinar la
c\'{a}mara, ya que si abrimos la c\'{a}mara sin rebobinarla, se nos van a velar las fotos, y no queremos eso,
\textquestiondown verdad?


\begin{lstlisting} [showspaces=false, numbers=left, breaklines=true]
(defrule rebobinar
    (eq NO (send [miCamara] get-REBOBINADA))
    (eq (send [miCamara] get-N_FOTOS) (send ?carrete get-N_FOTOS))
=>
    (send [miCamara] putt-REBOBINADA SI)
    (send [miCamara] putt-N_FOTOS 0)
    (send [miCamara] putt-PELICULA (make-instance [nil] of CARRETE))
    (send [miCamara] putt-CARGADA NO)
)
\end{lstlisting}

Esta regla, b\'{a}sicamente lo que hace es, si no est\'{a} rebobinada, y el n\'{u}mero de fotos realizada con la
c\'{a}mara, es igual al n\'{u}mero de fotos que se pueden hacer con ese carrete, la rebobinamos, reiniciamos el
contador de fotos de la c\'{a}mara y ponemos una pel\'{i}cula nueva, en este caso utilizamos 
\texttt{[nil]} para decir que es \texttt{null}. Y ponemos que no est\'{a} cargada.

Por \'{u}ltimo, queda cargar un nuevo carrete.

\begin{lstlisting} [showspaces=false, numbers=left, breaklines=true]
(defrule cargarPelicula
    (eq SI (send [miCamara] get-REBOBINADA))
    ?nombreInstancia <- (INSTANCE-NAME-TO-SYMBOL (send [miCamara] get-PELICULA))
    (eq ?nombreInstancia nil)
    (eq NO (send [miCamara] get-CARGADA))
=>
    (send [miCamara] putt-PELICULA (make-instance [miCarrete] of CARRETE_DE_DOS))
    (send [miCamara] putt-CARGADA SI)
)
\end{lstlisting}

Lo que hace esta regla es: si tenemos la c\'{a}mara rebobinada, y la pel\'{i}cula que tiene
la c\'{a}mara es \texttt{nil}, es decir, no tiene pel\'{i}cula, y no est\'{a} cargada, pues creamos
una nueva instancia de carrete, y se la asignamos a la c\'{a}mara, y ponemos que est\'{a} cargada.

Ya tenemos la c\'{a}mara lista para disparar.

\appendix
\chapter{Notas}
Hemos programado todo sin haberlo podido probar, ya que and\'{a}bamos justos de tiempo. De todas formas, como
se puede observar, todo lo que hemos hecho programado esta justificado, es decir, no ha sido programado al tunt\'{u}n.

La idea principal es esa, este c\'{o}digo es muy mejorable, por ejemplo haciendo que pueda trabajar con m\'{a}s de una 
c\'{a}mara, eliminando el slot CARGADA y SACAR ya que creemos que son completamente irrelevantes.

\end{document}
