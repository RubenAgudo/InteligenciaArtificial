\documentclass[12pt]{article}
%\documentclass[A4, 12pt]{scrreprt}
\usepackage[spanish]{babel}
\usepackage{fontspec}
\usepackage{listings}
%\usepackage{classicthesis}
%\usepackage{arsclassica}
\usepackage[hidelinks]{hyperref}

\setmainfont{Helvetica Neue}

%opening
\title{Ejercicios CLIPS}
\author{Rub\'{e}n Agudo\\
        Jonatan Gale\'{a}n \\
        I\~{n}igo Ochoa de Erive}

\begin{document}

\maketitle

\begin{abstract}
    Resoluci\'{o}n de los ejercicios propuestos de CLIPS
\end{abstract}

\section{Ejercicio 1}
Mirar soluci\'{o}n en el archivo adjunto.
\section{Ejercicio 2}
El orden de ejecuci\'{o}n de las reglas es el siguiente siguiendo la
estrategia en profundidad.
\begin{lstlisting}[breaklines=true, 
numbers=left, showspaces=false]
CLIPS> (run)
FIRE    1 regla-1: f-1
==> f-2     (elemento 2)
==> Activation 10     regla-1: f-2
FIRE    2 regla-1: f-2
==> f-3     (elemento 3)
==> Activation 10     regla-1: f-3
FIRE    3 regla-1: f-3
==> f-4     (elemento 4)
==> Activation 10     regla-1: f-4
FIRE    4 regla-1: f-4
==> f-5     (elemento 5)
==> Activation 10     regla-1: f-5
FIRE    5 regla-1: f-5
==> f-6     (elemento 6)
==> Activation 10     regla-1: f-6
FIRE    6 regla-1: f-6
==> f-7     (elemento 7)
==> Activation 10     regla-1: f-7
FIRE    7 regla-1: f-7
==> f-8     (elemento 8)
==> Activation 10     regla-1: f-8
FIRE    8 regla-1: f-8
==> f-9     (elemento 9)
==> Activation 10     regla-1: f-9
FIRE    9 regla-1: f-9
==> f-10    (elemento 10)
==> Activation 10     regla-1: f-10
FIRE   10 regla-1: f-10
==> f-11    (elemento 11)
==> Activation 10     regla-1: f-11
FIRE   11 regla-1: f-11
==> f-12    (elemento 12)
==> Activation 10     regla-1: f-12
FIRE   12 regla-1: f-12
==> f-13    (elemento 13)
==> Activation 10     regla-1: f-13
FIRE   13 regla-1: f-13
==> f-14    (elemento 14)
==> Activation 10     regla-1: f-14
FIRE   14 regla-1: f-14
==> f-15    (elemento 15)
==> Activation 10     regla-1: f-15
FIRE   15 regla-1: f-15
==> f-16    (elemento 16)
==> Activation 20     regla-2: f-16
==> Activation 10     regla-1: f-16
FIRE   16 regla-2: f-16
[PRCCODE4] Execution halted during the actions of defrule regla-2.
)
\end{lstlisting}

\section{Ejercicio 3}
El orden de ejecuci\'{o}n de las reglas es el siguiente siguiendo la
estrategia en profundidad.
\begin{lstlisting}[breaklines=true, 
numbers=left, showspaces=false]
CLIPS> (run)
FIRE    1 regla-1: f-1,f-1
==> f-2     (elemento 2)
==> Activation 10     regla-1: f-1,f-2
==> Activation 10     regla-1: f-2,f-1
==> Activation 10     regla-1: f-2,f-2
FIRE    2 regla-1: f-2,f-2
==> f-3     (elemento 4)
==> Activation 10     regla-1: f-2,f-3
==> Activation 10     regla-1: f-1,f-3
==> Activation 10     regla-1: f-3,f-1
==> Activation 10     regla-1: f-3,f-2
==> Activation 10     regla-1: f-3,f-3
FIRE    3 regla-1: f-3,f-3
==> f-4     (elemento 8)
==> Activation 10     regla-1: f-3,f-4
==> Activation 10     regla-1: f-2,f-4
==> Activation 10     regla-1: f-1,f-4
==> Activation 10     regla-1: f-4,f-1
==> Activation 10     regla-1: f-4,f-2
==> Activation 10     regla-1: f-4,f-3
==> Activation 10     regla-1: f-4,f-4
FIRE    4 regla-1: f-4,f-4
==> f-5     (elemento 16)
==> Activation 10     regla-1: f-4,f-5
==> Activation 10     regla-1: f-3,f-5
==> Activation 10     regla-1: f-2,f-5
==> Activation 10     regla-1: f-1,f-5
==> Activation 10     regla-1: f-5,f-1
==> Activation 10     regla-1: f-5,f-2
==> Activation 10     regla-1: f-5,f-3
==> Activation 10     regla-1: f-5,f-4
==> Activation 10     regla-1: f-5,f-5
FIRE    5 regla-1: f-5,f-5
==> f-6     (elemento 32)
==> Activation 20     regla-2: f-6
==> Activation 10     regla-1: f-5,f-6
==> Activation 10     regla-1: f-4,f-6
==> Activation 10     regla-1: f-3,f-6
==> Activation 10     regla-1: f-2,f-6
==> Activation 10     regla-1: f-1,f-6
==> Activation 10     regla-1: f-6,f-1
==> Activation 10     regla-1: f-6,f-2
==> Activation 10     regla-1: f-6,f-3
==> Activation 10     regla-1: f-6,f-4
==> Activation 10     regla-1: f-6,f-5
==> Activation 10     regla-1: f-6,f-6
FIRE    6 regla-2: f-6
[PRCCODE4] Execution halted during the actions of defrule regla-2.
\end{lstlisting}

\section{Ejercicio 4}
El orden de ejecuci\'{o}n de las reglas es el siguiente siguiendo la
estrategia en profundidad.
\begin{lstlisting}[breaklines=true, 
numbers=left, showspaces=false]
CLIPS> (run)
FIRE    1 regla1: f-3
==> f-4     (valor 5)
==> Activation 1      regla3: f-4
FIRE    2 regla1: f-2
==> f-5     (valor 8)
==> Activation 5      regla2: f-4,f-5
==> Activation 1      regla3: f-5
FIRE    3 regla1: f-1
==> f-6     (valor 1)
==> Activation 1      regla3: f-6
==> Activation 5      regla2: f-6,f-4
==> Activation 5      regla2: f-6,f-5
FIRE    4 regla2: f-6,f-5
<== f-6     (valor 1)
<== Activation 5      regla2: f-6,f-4
<== Activation 1      regla3: f-6
FIRE    5 regla2: f-4,f-5
<== f-4     (valor 5)
<== Activation 1      regla3: f-4
FIRE    6 regla3: f-5
Resultado: valor 8
<== f-5     (valor 8)
CLIPS>
\end{lstlisting}

\section{Ejercicio 5}
Nuestra soluci\'{o}n para el ejercicio del factorial es el siguiente, pero
no hemos utilizado la instrucci\'{o}n \texttt{salience} ya que no lo hemos
considerado oportuno. 

Para evitar los conflictos entre reglas, lo que hacemos
es a\~{n}adir una y borrar la anterior. En cada ejecuci\'{o}n de la regla
\texttt{factorial} restamos uno al n\'{u}mero introducido por el usuario, y 
acumulamos, multiplicando lo que hab\'{i}a por el ciclo actual.

El caso base, sirve tanto para parar, como para el caso del factorial de 0.
\begin{lstlisting}[breaklines=true, 
numbers=left, showspaces=false]
(defrule iniciar
    (initial-fact)
    =>
        (printout t "De que numero quieres hacer el factorial: ")
        (bind ?numero (read))
        (assert (numsFactorial ?numero 1))
)

(defrule factorial
    ?paBorrar<-(numsFactorial ?x ?y)
    (not (numsFactorial 0 ?y))
    =>
        (assert (numsFactorial (- ?x 1)(* ?y ?x)))
        (retract ?paBorrar)
)

(defrule casoBase
    ?paBorrar<-(numsFactorial 0 ?y)
    =>
        (printout t "Resultado: valor " ?y crlf)
        (retract ?paBorrar)

)
\end{lstlisting}

\end{document}
